\subsection{MS Visual Studio 2019 konfigurieren}
Sobald sichergestellt ist, dass obig aufgeführte Voraussetzungen gegeben sind, kann die IDE MS Visual Studio 2019
gestartet werden. Führe die folgenden Schritte durch:
\begin{itemize}
    \item Auf der rechten Seite siehst du diverse Optionen, um ein Projekt zu öffnen oder zu erstellen. Wähle dort
    'Lokalen Ordner öffnen' aus.
    \item Navigiere zu deinem Verzeichnis, in das du das Projekt gespeichert hast.
    \item Öffne den Projektornder (meistens 'learn\textunderscore with\textunderscore tello').
    \item Nun wird das Projekt geöffnet und die CMake-Generierung gestartet, welche eventuell fehlschlagen wird.
    Sollte dies der Fall sein, so ist die CMake-Version des Visual-Studios zu gering und wir müssen diese manuell
    einstellen. Öffne in dem Fall den Explorer und navigiere in deinen Projektordner. Dort siehst du die Datei 'CMakeSettings.json'.
    Öffne diese in einem Editor und füge die folgende Zeile hinzu:\\
    \begin{lstlisting}[language=json, basicstyle=\small]
    {
        ...
        "cmakeExecutable": "D:\\Prog\\CMake-3.18.2\\bin\\cmake.exe"
        ...
    }
    \end{lstlisting}
    \item Sobald diese Änderungen gemacht sind und du zurück ins Visual Studio wechselst, siehst du, dass CMake bereits
    die Generierung gestartet hat. CMake macht nun in einem ersten Schritt das Basissetup, um der IDE mitzuteilen,
    wie das Projekt aufgebaut und wie die 'targets' generiert werden sollen.
\end{itemize}
Sobald CMake fertig ist, kannst du nun die nachfolgend aufgeführten 'Startelemente' erstellen.
Um dies zu bewerkstelligen, verschaffst du dir zuerst einen kleinen Überblick der IDE. Im Zentrum steht das
Code-Fenster, welches den meisten Platz einnimmt. Auf der rechten Seite siehst du den Projektmappen-Explorer. Dort
werden die verschiedenen Verzeichnisse sowie der Code gelistet.
Oben kannst du einen grünen Play-Button sehen, daneben steht zu Beginn 'Startelement auswählen'. Klappe dieses auf
und du solltest mindestens folgende Startelemente sehen:
\begin{description}
    \item[00\textunderscore base\textunderscore module.dll] Beinhaltet Basiseinstellungen für die Drohne(n)
    \item[01\textunderscore keyboard\textunderscore module.dll] Die ersten paar Übungen.
    \item[01\textunderscore keyboard\textunderscore module\textunderscore solution.dll] Die Lösungen zu den Übungen.
    \item[99\textunderscore template.dll] Dies ist eine Vorlage für neue Module, das kann ignoriert werden.
    \item[app\textunderscore basic.exe] Die Hauptapplikation (Um diese zu starten, drücke den grünen Play-Button hinter dem Dropdown mit den 'targets')
    \item[app\textunderscore common\textunderscore video.dll] Eine Bibliothek mit Code, den Module verwenden können.
\end{description}
Um ein Startelement zu kompilieren, muss dieses ausgewählt werden. Danach drückst du den Play-Button, um den Build
zu starten.\\
Für den Anfang reicht es, wenn die folgenden Startelemente erstellt werden (Achtung, nach dem Build einer .dll wird ein
Fehler angezeigt, dass diese nicht ausgeführt werden kann. Das ist ganz normal, klicke den Fehler einfach weg und
schliesse die automatisch geöffnete Datei launch.vs.json ohne zu speichern):
\begin{itemize}
    \item 00\textunderscore base\textunderscore module.dll
    \item 01\textunderscore keyboard\textunderscore module.dll
    \item app\textunderscore common\textunderscore video.dll
    \item app\textunderscore basic.exe
\end{itemize}
Schliesse die nun geöffnete Anwendung.\\
Wenn die Builds allesamt ohne Fehler durchlaufen, dann kannst du deine Tello-EDU-Drohne starten.
Du solltest sie zuerst mit der offiziellen App kalibrieren.
Stelle sie danach auf den Boden, wo genug Platz drumherum und nach oben ist. Die Drohne sollte gelb blinken, das heisst,
sie wartet auf eine Verbindung. Wähle in den WLAN-Einstellungen das Tello-EDU WLAN aus.\\
Wähle nun das Startelement 'app\textunderscore basic' aus und drücke den Play-Button. Um deine Umgebung zu testen, drücke
den Button 'Take off' in der Applikation. Wenn die Drohne fliegt und du das Kamerabild der Drohne siehst, hast du alles richtig gemacht
und kannst nun mit den Übungen starten. Drücke den Knopf nochmals, um die Drohne zu landen und schliesse die Applikation.