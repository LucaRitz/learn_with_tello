\subsection{Übung 2 - Einmal mehr eine Addition}
Wir wollen nun noch einen Schritt weitergehen  und uns einen Teil der Arbeit der \gls{ALU} anschauen,
nämlich die Addition zweier Zahlen im binären System. Um den Einstieg zu vereinfachen, wollen wir uns zuerst mit
etwas auseinandersetzen, das wir bereits in der Unterstufe gelernt haben. Die schriftliche Addition im 10er-System.\\
Weisst du noch, wie das geht?\\
\begin{exerciseseries}[columns=1,solsubrule=\hrule]{}
    \begin{exercise}
        Addiere die Zahlen 25 und 2 schriftlich. Wenn du nicht mehr weisst, wie das geht, dann schaue in den Lösungen.
        \vspace{5cm}
    \end{exercise}
    \begin{solution}
        \opadd[carryadd=true,lastcarry]{25}{2}
    \end{solution}
    \begin{exercise}
        Addiere die Zahlen 25 und 7.
        \vspace{5cm}
    \end{exercise}
    \begin{solution}
        \opadd[carryadd=true,lastcarry]{25}{7}
    \end{solution}
    \begin{exercise}
        Addiere die Zahlen 99 und 99.
        \vspace{5cm}
    \end{exercise}
    \begin{solution}
        \opadd[carryadd=true,lastcarry]{99}{99}
    \end{solution}
    \begin{exercise}
        Addiere die Zahlen 45 und 72.
        \vspace{5cm}
    \end{exercise}
    \begin{solution}
        \opadd[carryadd=true,lastcarry]{45}{72}
    \end{solution}
\end{exerciseseries}
\newpage
Da wir jetzt ein wenig repetiert haben, können wir uns etwas Neuem widmen. Wir machen jetzt genau dasselbe, nur
im binären System.

\begin{exerciseseries}[columns=1,solsubrule=\hrule]{}
    \begin{exercise}
        Addiere die Zahlen 2 (10) und 1 (1) schriftlich im Binärsystem.
        \vspace{5cm}
    \end{exercise}
    \begin{solution}
        \opaddbin{2}{1}{}
    \end{solution}
    \begin{exercise}
        Addiere die Zahlen 3 (11) und 1 (1) schriftlich im Binärsystem.
        \vspace{5cm}
    \end{exercise}
    \begin{solution}
        \opaddbin{3}{1}{6}
    \end{solution}
\end{exerciseseries}

\newpage
\subsection{\solutionsname}
\loadSolutions