\newpage
\subsection{Übung 3 - Logische Operationen}
Was uns jetzt noch fehlt ist das Verständnis von logischen Operationen. Es geht im Grunde nur darum,
zwei Bedingungen auszuwerten und auf einen Wahrheitswert zu überführen. Nehmen wir als Beispiel einmal an,
dass wir die folgenden Aussagen haben:
\begin{description}
    \item[Aussage A] Ein Flugzeug ist zum Tauchen gebaut worden.
    \item[Aussage B] 9 ist durch 3 teilbar.
    \item[Aussage C] 5 ist eine Primzahl.
    \item[Aussage D] YB gewinnt in der nächsten Saison die Championsleague.
\end{description}
Wir haben offensichtlich zwei Aussagen (B und C), die der Wahrheit entsprechen und eine
Aussage (A), die nicht zutrifft. Weiterhin gibt es eine Aussage (D), bei der wir den Wahrheitswert nur
erahnen können, aber nicht genau kennen. Doch wir können festhalten, dass in der klassischen Aussagenlogik
eine Aussage also entweder wahr oder falsch sein kann.\cite{wikipedia:aussagenlogik}\\

Wir können diese Aussage nun verknüpfen und daraus eine Neue generieren. Dazu schauen wir uns zwei
Werkzeuge an:
\begin{itemize}
    \item AND (Und) $\land$
    \item OR (Oder) $\lor$
\end{itemize}
Angenommen, wir setzen die Aussagen B und C mithilfe des logischen AND zusammen, dann
lautet die neue Aussage:\\$B \land C$\glqq 9 ist durch 3 teilbar UND 5 ist eine Primzahl\flqq\\
Diese Aussage ist ebenfalls wahr.

\begin{exerciseseries}[columns=1,solsubrule=\hrule]{}
    Wir wollen nun mithilfe dieser Verknüpfungen ein paar Aussagen zusammensetzen. Kreuze das korrekte Resultat an.
    \begin{exercise}
        $A \land B$\\
        Wahr $\square$ \hspace{1cm} Falsch $\square$
    \end{exercise}
    \begin{solution}
        Wahr $\square$ \hspace{1cm} Falsch $\boxtimes$
    \end{solution}

    \begin{exercise}
        $A \lor B$\\
        Wahr $\square$ \hspace{1cm} Falsch $\square$
    \end{exercise}
    \begin{solution}
        Wahr $\boxtimes$ \hspace{1cm} Falsch $\square$
    \end{solution}

    \begin{exercise}
        $B \lor C$\\
        Wahr $\square$ \hspace{1cm} Falsch $\square$
    \end{exercise}
    \begin{solution}
        Wahr $\boxtimes$ \hspace{1cm} Falsch $\square$
    \end{solution}
\end{exerciseseries}

\newpage
Zuletzt wollen wir uns noch kurz ansehen, wie wir diese Operationen auch auf binären Zahlen durchführen können
und wie das Resultat aussieht. Dazu betrachten wir uns folgenden Zahlen, welche nun eine Aussage darstellen.
\begin{description}
    \item[Aussage A] 1101 $(13)_{10}$
    \item[Aussage B] 1111 $(15)_{10}$
\end{description}
Um diese Operation nun korrekt durchführen zu können, brauchen wir eine Wahrheitstabelle, um nachschauen zu können,
was bei bestimmten Kombinationen passiert.\\
Die Warheitstabelle für ein logisches AND sieht folgendermassen aus.
\[
    \begin{array}{c | c c}
        \land & 0 & 1 \\ \hline
        0 & 0 & 0 \\
        1 & 0 & 1 \\
    \end{array}
\]
Die Warheitstabelle für ein logisches OR sieht folgendermassen aus.
\[
    \begin{array}{c | c c}
        \lor & 0 & 1 \\ \hline
        0 & 0 & 1 \\
        1 & 1 & 1 \\
    \end{array}
\]
Wir führen nun ein logisches AND auf der Aussage A und B durch.
Dies sieht folgendermassen aus:
\opandbin{13}{15}{13}
Das Resultat dieser Operation ist also wieder 1101 $(13)_{10}$.
\newpage
\begin{exerciseseries}[columns=1,solsubrule=\hrule]{}
    Führe folgende logischen Operationen durch und bestimme auch gleich die Zahl im Zehnersystem.
    \begin{exercise}
        $11001 \land 00110$\\
        \vspace{5cm}
    \end{exercise}
    \begin{solution}
        \opandbin{25}{6}{0}
        Dies entspricht $(0)_{10}$.
    \end{solution}

    \begin{exercise}
        $11001 \lor 00110$\\
        \vspace{5cm}
    \end{exercise}
    \begin{solution}
        \opandbin{25}{6}{31}
        Dies entspricht $(31)_{10}$.
    \end{solution}

    \begin{exercise}
        $1111 \land 0110$\\
        \vspace{5cm}
    \end{exercise}
    \begin{solution}
        \opandbin{15}{6}{6}
        Dies entspricht $(6)_{10}$.
    \end{solution}
\end{exerciseseries}

\newpage
\subsection{\solutionsname}
\loadSolutions